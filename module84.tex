\setlength{\parindent}{0pt}
\clearpage

\section{More DFA Examples}

\subsection{Common Subexpression Elimination}


In this analysis, the property that we want to know at a program point is all the expressions available at that point.
\begin{itemize}
    \item The idea is to maintain a set of available expressions and the temporary in which they are stored, for every program point
    \item If a subexpression is available in the set of available expressions just before the statement that computes that subexpression, then replace it by the corresponding temporary (the precomputed value)
\end{itemize}

%Insert Examples

\subsection{Available expressions DFA}
In this analysis, the DFA value that we will deal with is a \underline{set of available expressions}, where each element in this set is a tuple of the register and the expression stored in that register.

So $(x, y+z)$ denotes that the value of the subexpression $y+z$ is stored in the temporary $x$.

\vspace{0.5cm}
Here also we will set a partial ordering in the values as follows:\\
$$s_2 \leqslant s_1 \textbf{ iff } s_2 \subseteq s_1$$

The lowest value in this ordering is $\{\}$, the empty set. This is a conservative value as it denotes that there is no Subexpression available, hence no subexpression available.

Once this ordering has been established, we can easily see that $glb(s_1,s_2) = s_1 \cap s_2$.

\subsection{Transfer Function for Available Expressions}
This happens to be a forward dataflow analysis, so we will have 2 types of rules:

\begin{itemize}
    \item \textbf{Case 1:} A statement $s$ has one or more predecessor program points. So in this case, the $in$ value of the statement $s$ is expressed as a function of $out$ values of the predecessor program points.
    \item \textbf{Case 2:} For a given statement $s$, the $out$ value is a function of the $in$ value of that statement.
\end{itemize}

Define $set_{in}(s)$ be the set of available expressions before the statement $s$, and $set_{out}$ be the set of available expressions after the statement $s$. The transfer function has the following rules:

\begin{itemize}
    %insert images
    \item If $s$ is $x := y + z$, then remove all the set elements that refer to $x$ (all the expressions stored in $x$ and using $x$) from $set_{in}(s)$. Then add $(x,y+z)$ to $set_{out}(s)$.
    \item For a statement $s$ and it's predecessors, $set_{in}(s) = glb\{set_{out}(p_i)~|~p_i~\in~{\tt predecessor}(s)\}$
\end{itemize}

Also if $s$ is the starting statement, then the boundary condition for this algorithm would be $set_{in}(s) = \{\}$.

\subsection{Copy Propagation}
Copy Propagation can be easily modelled as a DFA analysis very similar to Available Expressions DFA, except that we will be limiting ourselves to statements of the form $x := y$. The transformation logic will also be similar.\\

This optimisation creates oppurtunities for other global optimizations such that Global Constant Propagation and Liveness Analysis, so it works well in tandem with them.

